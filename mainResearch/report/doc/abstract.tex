\newpage

\centerline{\scalebox{1.5}{\textbf{Abstract}}}
\vspace{12pt}



In recent years, with the advancement of 3D computer graphics content, there is a growing demand for high-quality and abundant content. Consequently, procedural modeling, which allows shape editing through parameters, has been widely employed. However, the increase in the number of parameters for improved quality poses a challenge, leading to a concern about extended development time. 


To address this issue, this study proposes a modeling support system that combines model proposals using interactive genetic algorithms and interpolation through a two-dimentional user interface (UI). We utilize genetic algorithms for their versatility and expand the exploration range through an enhanced UI compared to conventional approaches. 


The proposed system performs exploration through three types of operations.
First, select three from six recommendation models. This reduces the burden on decision-making compared to traditional interactive genetic algorithms by providing fewer choices.Subsequently, interpolation is conducted by dragging a pointer on a two-dimensional UI. The interpolation operation enables exploration of areas beyond the recommendation models. Therefore, it can compensate for the reduced exploration space resulting from the fewer choices.Finally, a slight adjustment of parameters is performed as conventionally done. Since there are constraints on the interpolation operation within the search space, fine-tuning allows for a better reflection of the user's intentions. After these operations, new recommendation models are computed using genetic algorithms. This sequence of steps is repeated to assist users in modeling.


While the experiment and survey results confirm the usefulness of the UI, the effectiveness of genetic algorithms could not be conclusively established. Furthermore, the proposed system is expected to demonstrate its effectiveness when dealing with a large number of parameters during free-form modeling.


A future challenge involves refining the model proposal method and exploring its application to other procedural content generation techniques.