\newpage

\chapter{おわりに}

\section{結論}
本研究ではプロシージャルモデリングにおける操作性およびモデリング時間の改善のため,インタラクティブ遺伝的アルゴリズムにおけるモデル提案と,平面 UI による補間を用いたモデリング支援システムを提案した.従来のパラメータを個別で動かすのではなく,モデルの補間という形でパラメータ群を同時に操作するとともに,平面 UI によって3つのモデルから補間されることにより直感的な操作で広い空間の探索を行うことが可能になった.ユーザスタディの結果,定量評価では既存手法に比べ時間的優位性を見出すことはできなかったが,定性評価から,自由かつ多くのパラメータが存在するときには有用性は確認できた.一方でインタラクティブ遺伝的アルゴリズムの有効性を確認することはできなかった.
\section{今後の課題}
課題点は大きく2点あり,他のプロシージャルモデリングへの対応とモデル提案手法の検討である.今回は2種類のモデルを用いたが,さらに多くのパラメータを持つ場合や,ストロークによって入力が行われるもの,あるいは物理演算やアニメーションといった様々なものへの応用の検討が考えられる.またモデル提案手法の改善として,本研究で用いた IGA では有用性が確認できなかったため,高次元への対応をしたベイズ最適化\cite{平沼智之2021組合せ最適化におけるベイジアン最適化アルゴリズムを組み込んだ遺伝的アルゴリズムの提案}や機械学習といった方法によって,効率的なモデル提案を行う事が出来ればより高速なモデリングに繋がると考えられる.